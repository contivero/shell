\documentclass{article}

\usepackage[utf8]{inputenc}
\usepackage{inconsolata}
\usepackage{hyperref}
\hypersetup{pdfauthor={Cristian Adrián Ontivero}}
\usepackage{listings}
\lstset{% general command to set parameter(s)
basicstyle =\small,          % print whole listing small
keywordstyle=\ttfamily,
identifierstyle=,           % nothing happens
commentstyle=\color{white}, % white comments
stringstyle=\ttfamily,      % typewriter type for strings
showstringspaces=false     % no special string spaces
}

\newlength\tindent%
\setlength{\tindent}{\parindent}
\setlength{\parindent}{0pt}
\renewcommand{\indent}{\hspace*{\tindent}}

%Pagination stuff.
\setlength{\topmargin}{-.3 in}
\setlength{\oddsidemargin}{0in}
\setlength{\evensidemargin}{0in}
\setlength{\textheight}{9.in}
\setlength{\textwidth}{6.5in}
\pagestyle{empty}

\begin{document}

\begin{center}
\Large Bash to POSIX shell cheatsheet
\end{center}

We will leave aside whether it is better to aim for POSIX compliance, or a
broader sense of portability (since some POSIX features aren't available
everywhere). One of the best ways to check for POSIX compliance is using
shellcheck on a script with the shebang {\tt \#!/bin/sh}. Unfortunately, the
documentation doesn't cover how to translate every ``bashism''. What follows is
in no way exhaustive.

Take only the first letter (Haskell's {\tt head}):

\begin{minipage}{.5\textwidth}
\begin{lstlisting}[language=sh]
"${string:0:1}"
\end{lstlisting}
\end{minipage}%
\begin{minipage}{.5\textwidth}
\begin{lstlisting}[language=bash]
  "$(echo "$string" | cut -c1)"
\end{lstlisting}
\end{minipage}

Take from the second letter onwards (Haskell's {\tt tail}):

\begin{minipage}{.5\textwidth}
\begin{lstlisting}[language=sh]
"${string:1:}"
\end{lstlisting}
\end{minipage}%
\begin{minipage}{.5\textwidth}
\begin{lstlisting}[language=bash]
  "$(echo "$string" | cut -c2-)"
\end{lstlisting}
\end{minipage}

To take a substring, say the first three letters:

\begin{minipage}{.5\textwidth}
\begin{lstlisting}[language=sh]
"${string:1:}"
\end{lstlisting}
\end{minipage}%
\begin{minipage}{.5\textwidth}
\begin{lstlisting}[language=bash]
"$(echo "$string" | cut -c1-4)"
\end{lstlisting}
\end{minipage}

Take the last letter (Haskell's {\tt last}):

\begin{minipage}{.5\textwidth}
\begin{lstlisting}[language=sh]
"${string:-1}"
\end{lstlisting}
\end{minipage}%
\begin{minipage}{.5\textwidth}
\begin{lstlisting}[language=bash]
"$(printf %s "$string" | tail -c 1)"
\end{lstlisting}
\end{minipage}

POSIX's {\tt echo} doesn't take any flags. In contrast, Bash's {\tt echo}
builtin does. To disable the newline, we can use {\tt echo -n} in Bash. The
portable POSIX way is using {\tt printf}:

\begin{minipage}[b]{.5\textwidth}
\begin{lstlisting}[language=sh]
echo -n 'No trailing newline'
\end{lstlisting}
\end{minipage}%
\begin{minipage}[b]{.5\textwidth}
\begin{lstlisting}[language=bash]
  printf %s 'No trailing newline'
\end{lstlisting}
\end{minipage}

In Bash, the {\tt -e} flags enables the interpretation of backslash escapes.
What's more, even among some of the main shells the defaults difer, for instance
Bash's {\tt echo} doesn't interpret escapes by default, but Zsh does. It is
preferable to use {\tt printf} as a portable alternative. When we want to
interpret escapes we use {\tt \%b}; when we don't, {\tt \%s}.

\begin{minipage}[b]{.5\textwidth}
\begin{lstlisting}[language=sh]
echo -e "Some\n\nText"
echo -E "Some\n\nText"
\end{lstlisting}
\end{minipage}%
\begin{minipage}[b]{.5\textwidth}
\begin{lstlisting}[language=bash]
printf '%b\n' -e "Some\n\nText"
printf '%s\n' -E "Some\n\nText"
\end{lstlisting}
\end{minipage}

Bash's {\tt read} builtin has the {\tt -s} flag which enables ``Silent Mode'': whatever is
input by the user isn't displayed. This is particularly useful when reading a
password. The {\tt -s} flag isn't part of the POSIX standard, but the same
effect can be achieved using {\tt stty}, which is part of POSIX:

\begin{minipage}[b]{.5\textwidth}
\begin{lstlisting}[language=sh]
read -s password
\end{lstlisting}
\end{minipage}%
\begin{minipage}[b]{.5\textwidth}
\begin{lstlisting}[language=bash]
stty -echo
read -s password
stty echo
\end{lstlisting}
\end{minipage}

Bash's {\tt read} builtin has the {\tt -p} flag, which displays a
\textit{prompt} (without a trailing newline), before reading input. The same can
be achieved in POSIX by simply printing before calling {\tt read}:

\begin{minipage}[t]{.5\textwidth}
\begin{lstlisting}[language=sh]
read -p 'Continue? [y/n] ' yn
\end{lstlisting}
\end{minipage}%
\begin{minipage}[t]{.5\textwidth}
\begin{lstlisting}[language=bash]
printf %s 'Continue? [y/n] '
read yn
\end{lstlisting}
\end{minipage}

POSIX compatible utility functions

\begin{lstlisting}[language=bash]
to_upper() { echo "$1" | tr '[:lower:]' '[:upper:]'; }

case_insensitive_match() {
  if [ "$(to_upper "$1")" = "$(to_upper "$2")" ]; then
    return 0
  fi
  return 1
}

str_head() { echo "$1" | cut -c1; }

str_tail() { echo "$1" | cut -c2- ; }

capitalize() {
  echo "$(to_upper "$(str_head "$first_word")")$(str_tail "$first_word")"
}

nub () { printf %b "$1" | sort | uniq ; }
\end{lstlisting}

\newpage 
\bibliography{refs}
\bibliographystyle{unsrt}

\end{document}


